\section{Introdução}
O objetivo do nosso projeto é criar um identificador linguístico capaz de determinar o idioma em que um texto específico está escrito, tendo como base alguns textos nas línguas que iremos verificar. Para isso, desenvolvemos um programa em Python que é capaz de acessar os textos bases e identificar a língua em que um novo texto está escrito.

A idéia é utilizar conceitos de \emph{Machine Learning}, apresentados em sala de aula, em nosso programa para maximizar o potencial do programa e aumentar o número de acertos na identificação dos textos. Para tornar essa análise gramatical possível, em um curto período de tempo, utilizamos inicialmente um algoritmo envolvendo a distância Euclidiana entre a porcentagem da frequência em que cada letra aparece. Por exemplo, encontramos a frequência em que a letra “a” aparece nos textos base (em português e inglês, nosso caso) e depois no texto que desejamos identificar. Assim, estabelecemos uma característica para cada texto.

\nocite{iaml:11}
